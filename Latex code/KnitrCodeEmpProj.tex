\documentclass[12pt,a4paper]{article}      % document type

\usepackage[margin=1in]{geometry}          % customize page layout (the text is large: 14.7cm, 418.25pt, 41.83em ) [top=8em,bottom=8em]
\usepackage{setspace}                      % allow line spacing
\setstretch{1.5}                           % line spacing
\setlength\parindent{0pt}                  % line indentation

\usepackage[T1]{fontenc}                   % 8-bit font encoding
\usepackage{xcolor}                        % add colors
\usepackage{moresize}                      % add \ssmall and \HUGE
\usepackage{mathpazo}                      % palatino font
%\usepackage{charter}                      % charter font
\usepackage{microtype}                     % improve general appearance
\usepackage{booktabs}                      % improve tables quality
\widowpenalty 10000                        % avoid widows
\clubpenalty 10000                         % avoid orphans

\usepackage{amsmath}                       % mathematical typesetting
\usepackage{amssymb, latexsym}             % import math symbols
\usepackage{textcomp}                      % import text symbols
\usepackage{enumerate}                     % allow enumerate counter styles
\usepackage{graphicx}                      % manage pictures
\usepackage{subfig}                        % allow subfigures
\usepackage{adjustbox}                     % add macros to adjust boxed content
\usepackage{tikz, tikz-qtree}              % create graphic elements
\usepackage{pdflscape}                     % allow landscape mode 
\usepackage[hidelinks,
            bookmarks=FALSE]{hyperref}     % use hyperlinks
\usepackage{url}                           % write url with all their characters
\usepackage{natbib}                        % bibliografy support (use \citet{key} or \citep{key})
%\usepackage{apacite}                      % Americ.Psych.Assoc. citations style
\renewcommand{\bibname}{References}        % choose name for reference section

%------------------------------------------------------------------%
\begin{document}
%------------------------------------------------------------------%

\pagenumbering{gobble}  % avoid numbering of cover page

%\title{Econometrics of Program Evaluation}
%\maketitle

\vspace{1cm}
\begin{center}
\text{\Large{Toulouse School of Economics}} \\
\vspace{2cm}
\text{\large{Empirical Project}} \\  % Econometric Production Analysis and Efficiency Analysis
\vspace{1cm}
\textbf{\Large{An Empirical Application of Stochastic Frontier Analysis in the Healthcare System}} \\
\vspace{1cm}
Jose Alvarez, Nicola Benigni, Irina Cotovici, Igor Custodio Jo\~{a}o \\
\vspace{1cm}
Supervised by: \\
Prof. Catherine Cazals \\
\vspace{1cm}
March 22, 2017
\end{center}

\vfill

\begin{abstract}
\noindent
The abstract goes here
\end{abstract}

%------------------------------------------------------------------%
\newpage 
\pagenumbering{arabic}
\tableofcontents
%\listoffigures
%\listoftables
%------------------------------------------------------------------%

\newpage
\section{Introduction}
\subsection{Literature review}

The national health care system of a country plays a fundamental role in its economy. One can reasonably expect, for example, that a healthy working population will be more productive throughout its active years than an unhealthy one, as the former is less likely to fall sick than the latter. It should not come as a surprise that the healthiest countries also tend to have the "healthiest" economies. Therefore, it is in a country's best interest to achieve a functioning and efficient health care system. The two concepts are not necessarily interchangeable: all efficient health care systems are indeed functional, but not all functioning health care systems are efficient. This paper explores the latter point. 
Using the World Health Organization's panel data on national care systems, we carry out first an econometric production analysis and then an efficiency analysis of these countries' national health care systems. The paper is organized in two parts. In \textbf{Part I}, we develop a model specification for the production function of the panel data and carry out a production analysis. In \textbf{Part II}, we proceed to do an efficiency analysis of the panel data.

\section*{Part I}

In its simplest form, we can define a production function for firm $i$ as
$$
\begin{aligned}
y_i = f(x_i)
\end{aligned}
$$

where $y_i$ is a given output, $x_i$ is the input (or the collection of inputs) needed for producing the given output, and $f(.)$ is the function that defines the relationship between the given output and its input(s). The idea can be easily extended to the context of this paper: the national health care system of a country is the output and national investment per capita on health infrastructure, for example, is one of the inputs. Similarly, $f(.)$ would determine how exactly that investment on health infrastructure translates into a better health care system. In this case rather than having firm $i$, we have country $i$.

Following Schmidt and Sickles (1984) and Greene (2004), we extend the above equation to account for the panel data nature of the sample and denote the production function as
$$
\begin{aligned}
y_i = f(x_i) + v_{it}, \quad \textrm{where} \quad v_{it} = \alpha_i + \epsilon_{it}
\end{aligned}
$$

where both input(s) and output are indexed by country $i$ and time $t$. The composite error term, $v_{it}$, consists of two elements: the idiosyncratic errors term (i.e. an exogenous random shock) and the time-invariant, country-specific characteristic $\alpha_i$, which captures unobserved heterogeneity. Both are unobserved to the econometrician, but it is $\alpha_i$ that threatens the estimation of the model. In the above equation, we assume $\mathbb{E}[x' \epsilon]=0$ to complete our model. Regarding $\alpha_i$ we can tackle it in a Fixed Effects framework, where $\mathbb{E}(x' \alpha) \neq 0$, or in a Random Effects framework, where $\mathbb{E}(x' \alpha) = 0$. Both approaches can only be done in a panel data setting. In the case of cross sectional analysis, there exists the risk that unobserved country-specific heterogeneity is creating measurement error and thus biased estimators. Greater details will be provided later in the model specification section. 

In Greene (2004), $\alpha_i$ is treated as a measure of a country's inefficiency that is unobserved to the econometrician. The idea behind it is that only country $i$ knows exactly how efficient or inefficient it is. Greene argues that what might seem unobserved heterogeneity might actually be country-specific inefficiency. Greene goes beyond the traditional Fixed Effect and Random Effect models by using amore flexible model, which we refer to as the Greene Approach'as so gar only his paper uses it. Although we do not use the Greene Approach, it is important to keep in mind Greene's point that unobserved heterogeneity might simply be unmeasured inefficiency.

So far, we have not specified the functional form of the production function, $f(.)$. The standard production function involves a parametrically defined family of functions. Henningsen (2014) provides a good survey of these families of functions, the most popular one being the Cobb-Douglas production function. Less traditional production functions' or, in other words, less restrictive models' can also be used in estimating production functions (Afriat, 1972). One these approaches is using nonparametric estimation methods. This paper focuses on the traditional approach, although nonparametric techniques are used to justify restrictive assumptions, such as assuming a linear functional form in the production function.

Part I is organised as follows...


\subsection{Theoretical framework}
Cobb-Douglas

Transcendental logarithmic 

Constant substitution elasticity

Stochastic Frontier Analysis



\section{Data}





































































